\chapter{緒論}
%\label{c:intro}
\section{研究背景} 

近年因為圍棋AI—AlphaGo,戰勝了人類職業選手,吸引越來越多人投入到人工智慧這塊領域。人工智慧這個主題其實已經有了很多年的歷史,但是直到最近這幾年才逐漸浮出檯面並受到重視,除了硬體及各項技術的進展,主要也因為人工智慧需要大量的數據做「訓練」。對早期研究者來說,想要獲得不錯效果的最小量測試,都遠遠超過當時的計算能力以及可以取得的數據資料量,也就是要取得大量的測試數據非常不容易。但最近幾年,由於網際網路和各項技術的進展,越來越多大型的資料庫出現,並且網路上有各式各樣的數據可以取得,許多研究人員重新挖掘神經網路的價值,這其實也是透過「大數據」使得架構的模型和測試都更有效率。\cite{name1}

在另一方面,由於手機和網際網路的快速發展,一種新的網頁應用技術稱為 Progressive Web Apps (PWA) 也應運而生。目前大部份使用於手機或電腦的應用程式,都要先透過安裝才可以使用,但是 PWA 只要能進行網頁瀏覽就可以使用,完全不需要進行軟體安裝,而使用起來的感覺與一般應用程式幾乎沒有什麼區別。PWA 可以說是結合了網頁和應用程式的使用者體驗,可以直接在使用者的裝置上瀏覽和執行,不需要透過應用程式商店下載和安裝。PWA 應用不僅可以提供精彩的全螢幕體驗,甚至可以透過網頁推播通知,吸引使用者繼續參與互動\cite{name3}。

PWA 的興起與網頁前端技術的快速發展有密切關係,前端網頁使用 HTML、XHTML、HTML5、CSS、JavaScript等網頁標準技術製作,與後端伺服器所使用的 PHP、ASP.NET、JSP、RoR 等程式語言有很大的不同。前端網頁的技術發展很快,可以用來製作各種 PWA 應用,目前最常見的三大架構為 AngularJS、React 和 VueJS。使用前端網頁技術,可以快速開發網頁的整體面貌,並可以利用非同步方式存取網頁所要呈現的內容,而開發過程隨時修改,隨時都可以看到更新的畫面結果,可以加速網頁應用的製作開發過程。\cite{name2}

\section{研究動機與目的}

近幾年我的指導教授所開授的程式課程,大多使用線上評測系統「瘋狂程設」\cite{name4} 做為輔助教學平台。「瘋狂程設」主要是謝育平教授與其學生所創作的程式教學網站。\cite{name5}。這個系統可以供教師開課,讓修課學生上線練習解題。系統具有多項功能,在教學講解、作業、以及考試等各方面都提供了相當不錯的功能。另外在評測方面,可以即時提供同學編譯上的錯誤,以及執行結果與正確答案之間的差異,可以說在教學上提供了很多的幫助。雖然此評測系統十分優異,但長期使用之後,仍然會感受到一些限制和不足之處。假如這個系統可以具備一些人工智能,例如自動分析同學常犯的錯誤,評估同學學習的狀況等,那麼這個系統就可以變得更為強大,對教學應該也可以提供更多的幫助。

因此我們著手進行這個研究,希望從作答過程一直到評測結果,可以透過系統自動蒐集一些數據,並可以用這些數據做進一步的分析,未來在這個基礎之上,就有機會可以打造一個融入人工智能的程式解題平台。由於瘋狂程設是別人的作品,而且沒有提供其原始碼,因此我們先在一個雛形平台上進行研究,這個雛形平台目前由指導教師和另一位同學周身鴻進行研究開發,在此也一併表達感謝。我們希望在這個平台上,新增一個即時監控的子系統,可以即時監控同學的作答過程,並進行一些數據的統計和分析,並且希望透過這些取得的數據,可以模擬同學作答的過程。未來希望在這個基礎之上,可以增加更多的人工智慧,讓這個系統變得更加強大。

\section{研究流程規劃}

以下是本論文的研究構想以及流程規劃:

\begin{itemize}%项目符号开始
	\item 研究構想
		\begin{enumerate}[1.]%编号开始
			\item 即時線上監測:測試的平台主要以網頁做為最主要和使用者互動的介面,其頁面設計採用網頁前端技術製作,我們希望利用相關的技術擷取同學作答過程的取樣資料,並能將這些資料存放到雲端的資料庫中,作為未來進一步分析的基礎。
			\item 基礎數據分析:透過擷取到的抽樣資料,希望可以分析同學作答過程的一些統計資訊,也希望可以即時模擬同學作答的畫面,以及由抽樣數據產出模擬同學作答過程的影片。
		\end{enumerate}

	\item 流程規劃
		\begin{enumerate}[1.]
			\item 訂定研究主題
			\item 決定研究目的與範圍
			\item 背景技術討論與資料蒐集
			\item 選擇開發環境
			\item 實驗兩部分程式並且做出結果表格
			\item 結果分析與討論
			\item 結論與未來發展
		\end{enumerate}
\end{itemize}
\newpage
\section{章節概要} 
\begin{itemize}
	\item 第一章
		\begin{itemize}
			\item 論文緒論
			\item 研究背景、動機與目的及章節概述
		\end{itemize}
	\item 第二章
		\begin{itemize}
			\item 背景技術介紹
			\item 說明研究所需之背景知識
		\end{itemize}
	\item 第三章
		\begin{itemize}
			\item 系統架構與規劃
			\item 介紹整個系統架構與所使用之語法邏輯解說
			\item 系統前半js與html之偵測
			\item 系統後半分析部分
		\end{itemize}
	\item 第四章
		\begin{itemize}
			\item 系統細部與執行結果
			\item 多加解釋細部活動
			\item 分析部分之結果呈現
	\end{itemize}
	\item 第五章
		\begin{itemize}
			\item 結果與未來展望
			\item 說明未來方向與檢討
		\end{itemize}
\end{itemize}