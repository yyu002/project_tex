\chapter{緒論}
\label{c:intro}
\section{研究背景} 

近年因為神經網絡圍棋AI──AlphaGo,戰勝了人類職業選手,越來越多人投入到人工智慧這塊領域。雖然神經網絡在幾十年前就有了,直到最近才浮出檯面。這是因為他們需要大量的「訓練」去發現矩陣中的數字價值。對早期研究者來說,想要獲得不錯效果的最小量測試,都遠遠超過計算能力和能提供的數據的大小。但最近幾年,一些能獲取大量資源的團隊重現挖掘神經網絡,其實就是透過「大數據」來使測試更有效率。\cite{name1}

安裝於手機或是電腦亦或其他行動裝置的應用app,如在瀏覽器上達成同樣效果,而不需進行安裝動作,可謂網頁app,以下稱web app。前端網頁使用 HTML / XHTML / HTML5 + CSS + Java Script … 等網頁標準技術製作,後端使用 PHP 、 ASP.NET 、 JSP 、 RoR … 等程式語言開發,並連結資料庫或其它資料來源,且透過瀏覽器輸入網址後執行。 Web App 只要使用裝置的瀏覽器輸入網址即可執行測試。若有任何問題,程式修改後,可以快速的進行測試,甚至有時只需要簡單的重新整理網頁即可。\cite{name2}

Progressive Web Apps(PWA)就是結合網頁和應用程式的使用者體驗。PWA可以直接在使用者的裝置主螢幕上安裝與執行,不需要透過應用程式商店取得。因為有網路應用程式資訊清單檔,所以 PWA 可以提供精彩的全螢幕體驗,甚至可以透過網頁推播通知,吸引使用者繼續參與互動。\cite{name3}


\section{研究動機與目的}

本校電機系開設的計算機概論課程所使用的線上評測系統是瘋狂程設\cite{name4}此系統為謝育平老師與其學生之原創網站。\cite{name5}

鑒於雖此評測系統十分優異,但使用了長時間之後些許感受到此系統有些許地方不足夠,或是希望此系統多一些應用等等。
而且若要使用此系統的程式評測,必須下載桌面版方能使用此評測系統,而網頁版只供使用者查詢或是紀錄查看。
在紀錄部分,正是我想所討論與擴展處,系統僅僅只是把每位同學的作答狀況記錄下;比如說只包含每人的作答時間與全部字數等,並且計算評分。

而我們想知道的是,同學在作答中的行為,想分析個人的作答狀況;在本系統可能無法看到太過細部的東西。
因此設計一新的系統,可以在線使用,透過瀏覽器傳遞,時間間隔更小的分析並記錄使用者的狀況。
類似一種簡單的爬蟲程式。\cite{name6}

\section{研究流程規劃}

因為想從瘋狂程設此系統發想,首先要做的是為觀察原本的系統:已經做到哪些,以及想要加上哪些功能等等。

\begin{itemize}%项目符号开始
	\item 本研究認為最主要核心為兩處
		\begin{enumerate}[1.]%编号开始
			\item 線上式觀察:
			因認為桌面板操作的普及與便利性不及使用瀏覽器的網頁版本,新的系統架設在網頁端才會更加方便。
			\item 分析數據:
			原系統只是記錄完成後的最終數據,我們若想知曉更加細微的地方可能無從所知。因此結合網頁與分析,希望同學們一邊作答時,後面網頁的程式一邊運作,如此便可以知道不同於繳交的數據,還知道作答中的細部數據。
		\end{enumerate}

	\item 規劃
		\begin{enumerate}[1.]
			\item 訂定研究主題
			\item 決定研究目的與範圍
			\item 背景技術討論與資料蒐集
			\item 選擇開發環境
			\item 實驗兩部分程式並且做出結果表格
			\item 結果分析與討論
			\item 結論與未來發展
		\end{enumerate}
\end{itemize}
\section{章節概要} 
\begin{itemize}
	\item 第一章
		\begin{itemize}
			\item 論文緒論
			\item 研究背景、動機與目的及章節概述
		\end{itemize}
	\item 第二章
		\begin{itemize}
			\item 背景技術介紹
			\item 說明研究所需之背景知識
		\end{itemize}
	\item 第三章
		\begin{itemize}
			\item 系統架構與規劃
			\item 介紹整個系統架構與所使用之語法邏輯解說
			\item 系統前半js與html之偵測
			\item 系統後半分析部分
		\end{itemize}
	\item 第四章
		\begin{itemize}
			\item 系統細部與執行結果
			\item 多加解釋細部活動
			\item 分析部分之結果呈現
	\end{itemize}
	\item 第五章
		\begin{itemize}
			\item 結果與未來展望
			\item 說明未來方向與檢討
		\end{itemize}
\end{itemize}